\chapter{Queue Results}
\label{app:queues}

All experiments are run on a 100GB DRAM machine with two 6-core Intel Xeon X5690 processors clocked at 3.47GHz. Hyperthreading is enabled in each processor, resulting in 24 available logical cores. The machine runs a 64-bit Linux 3.2.0 operating system, and all benchmarks and STO data structures are compiled with \texttt{g++-5.3}. In all tests, threads are pinned to cores, with at most one thread per logical core.
In all performance graphs, we show the median of 5 consecutive runs with the minimum and maximum performance results represented as error bars.

Cache misses are recorded by running the Multi-Thread Singletons Test benchmark with 8 threads, with each thread performing 10M transactions, under the profiling tool Performance Events for Linux (\texttt{perf}). The sampling period is set to 1000, meaning that every 1000th cache miss is recorded.
We report the number of cache misses reported by perf (approximately 1/1000 of the actual number of cache misses).

\section{Cache Misses}
    \label{app:qcm}
\begin{figure}[H]
    \centering
    \includegraphics[width=\textwidth]{fcqueues/cm.png}
    \caption*{Queue Cache Misses}
\end{figure}

\section{Performance of Non-transactional Concurrent Queues}


\begin{figure}[H]
    \centering
    \includegraphics[width=0.85\textwidth]{concurrent/allQ:RandSingleOps10000.png}
    
    \vspace{12pt}
    \includegraphics[width=0.85\textwidth]{concurrent/allQ:RandSingleOps100000.png}
    \caption*{Multi-Thread Singletons Test}
\end{figure}
\begin{figure}[H]
    \centering
    \includegraphics[width=\textwidth]{concurrent/allQ:PushPop.png}
    \caption*{Push-Pop Test (2 threads)}
\end{figure}

\section{Performance of Various Transactional Queues}
\begin{figure}[H]
    \centering
    \includegraphics[width=0.85\textwidth]{fcqueues/allQ:RandSingleOps10000.png}
    \vspace{20pt}
    \includegraphics[width=0.85\textwidth]{fcqueues/allQ:RandSingleOps100000.png}
    \caption*{Multi-Thread Singletons Test}
\end{figure}
\begin{figure}[H]
    \centering
    \includegraphics[width=\textwidth]{fcqueues/allQ:PushPop.png}
    \caption*{Push-Pop Test (2 threads)}
\end{figure}

\section{Push-Pop Test: Ratio of Pops to Pushes}

\begin{table}[H]
        \centering
    \begin{tabular}{|cc|}
        \hline
        Queue & Pops per 100 Pushes\\
        \hline
            T-QueueO & 64\\
            T-QueueP & 28\\
            T-FCQueue & 57\\
            NT-FCQueueWrapped & 94\\
            NT-FCQueue & 110\\
            Basket & 42\\
            Moir & 62\\
            Michael-Scott& 54\\
            Optimistic & 50\\
            Read-Write & 84\\
            Segmented & 50\\
            TsigasCycle & 52\\
        \hline
    \end{tabular}
    \caption*{Ratio of pops to pushes when the push-only and pop-only threads are executing simultaneously}
\end{table}


\section{Abort Rate Results}
\label{app:queue_mt}

\begin{table}[H]
\begin{figure}[H]
    \centering
    \begin{tabular}{|c|c|c|}
\hline
\multirow{2}{*}{Queue} & \multicolumn{2}{c|}{Initial Size}\\\cline{2-3}& \qquad 10000 \qquad\quad & \qquad 100000\qquad\quad\\
\hline
\hline
T-QueueO & 0.00001 & 0.00001\\
T-QueueP & 1.54560 & 1.60886\\
T-FCQueue & 0.62494 & 1.62620\\
WT-FCQueue & 0.00000 & 0.00000\\
\hline\end{tabular}

    \caption*{Push-Pop Test (2 threads)}
\end{figure}
\begin{figure}[H]
    \centering
        \begin{tabular}{|c|c|c|c|}
\hline
\multirow{2}{*}{Queue} & \multicolumn{3}{c|}{\#Threads}\\\cline{2-4}& \quad 4 & 12 & 20\\
\hline
\hline
T-QueueO & 9.55031 & 6.77475 & 6.38129\\
T-QueueP & 2.87497 & 2.29600 & 2.46225\\
WT-Queue & 17.01557 & 57.89091 & 71.00616\\
NT-FCQueueWrapped & 0.00000 & 0.00000 & 0.00000\\
T-FCQueue & 4.62935 & 3.46799 & 4.28826\\
WT-FCQueue & 0.00000 & 0.00000 & 0.00000\\
\hline\end{tabular}

    \caption*{Multi-Thread Singletons Test: Initial Size 10000$^*$}
\end{figure}
\begin{figure}[H]
    \centering
    \begin{tabular}{|c|c|c|c|c|c|c|}
\hline
\multirow{2}{*}{Queue} & \multicolumn{6}{c|}{\#Threads}\\\cline{2-7}& 2 & 4 & 8 & 12 & 16 & 20\\
\hline
\hline
T-QueueO & 7.88767 & 10.67091 & 7.85545 & 7.54535 & 6.73918 & 6.48418\\
T-QueueP & 3.07026 & 2.70427 & 2.12894 & 2.22341 & 2.22941 & 2.27197\\
T-FCQueue & 4.05484 & 5.98340 & 4.51612 & 4.83815 & 5.08319 & 5.22679\\
WT-FCQueue & 0.00000 & 0.00000 & 0.00000 & 0.00000 & 0.00000 & 0.00000\\
\hline\end{tabular}

    \caption*{Multi-Thread Singletons Test: Initial Size 100000$^*$}
\end{figure}
    \caption*{\footnotesize{$^*$We note that the abort rate appears to spike at 4 threads and decrease as the number of threads increases. One possible explanation may be that contention varies due to the spread of the threads among the CPU sockets.}}
\end{table}
