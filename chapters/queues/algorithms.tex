\chapter{FIFO Queue Algorithms and Analysis}
\label{queue}

This chapter investigates different concurrent and transactional algorithms for queues to draw conclusions about concurrent queue algorithms in transactional settings. We begin with an overview of concurrent and transactional queue specifications and algorithms. We then evaluate how these queues perform on several microbenchmarks. Given our results, we conjecture that highly-concurrent queue algorithms are inherently unsuited to be converted for use as a fully transactional queue algorithm. The optimizations taken by these algorithms rely on data structure state and behavior that must be modified to support transactions. In other words, the highly-concurrent queue algorithm's synchronization mechanism interferes with the mechanisms that STO uses to provide transactional guarantees.

\section[Transactional Queue Specification]{Transactional Queue Specification
\footnote{In the following discussion of our queue algorithms, we omit the discussion of the front operation to simplify reasoning about the state of the queue. An appropriate algorithm for front can be easily inferred from those used for pop.}}

A concurrent queue supporting operations push and pop must adhere to the following specification:
\begin{itemize}
    \item A value is popped off the queue only once (no duplicate pops).
    \item A value is pushed onto the queue only once (no duplicate pushes).
    \item Values are popped in the order in which they are pushed.
\end{itemize}

\noindent
A transactional queue adds the following invariants to the specification. There must be a serial order of all transactions such that, within one transaction:
\begin{itemize}
    \item Any two pops pop consecutive values in the queue starting from the head of the queue.
    \item Any two pushes push consecutive values at the tail of the queue.
\end{itemize}

\noindent
To satisfy these invariants, transactional data structures must support \emph{read-my-writes}. This is when a thread sees and modifies or returns the value from a previous operation in the transaction.


%%%%%%%%%%%%%%%%%%%%%%%%%%%%%%%%%%%%%%%%%%%%%%%%%%%%%%%%%%%%%%%%%%%%%%%%%%%%%%%%%%%%%%%%%%%%%%%%%%
\section{Naive Synchronization Queue Algorithms}

STO provides two transactional FIFO queues that adhere to the interface exposed by the \texttt{C++} standard library queue. These transactional queue algorithms are designed with transactional correctness in mind, and concurrency as a secondary concern. 

These two algorithms enforce transactional correctness using \emph{versions}. A version can act as a lock on the data structure: in order to update the data structure, a thread must first lock the version. A version also tracks changes to the data structure because it monotonically increases when a thread modifies the data structure. Thus, any version seen by a thread is equivalent to some previous or current state of the data structure. The first instance of the version observed by a thread during a transaction is checked when the transaction commits. This ensures that all observations are valid. Note that we cannot update the read version to an instance of the version observed later in the transaction. This is because we need to validate the first time we see the version in the transaction. 

\subsection{T-Queue1}
The T-Queue1 is the first implementation of the transactional queue data structure using STO's framework. It implements a circular, fixed-size transactional queue.

The queue supports transactional operations push and pop and is implemented using optimistic concurrency control (OCC). This means that two threads can simultaneously access the queue during their transaction's executions. At commit time, the threads check if the queue has changed in a way that would invalidate their transactions. The T-Queue1 allows checks on the queues to be done via two versions: a headversion to check if another thread has popped from the queue, and a tailversion to check if another thread has pushed onto the queue. The headversion tracks the location of the head of the queue, and the tailversion tracks the location of the tail of the queue.

A transactional push adds to an internal \texttt{write\_list}, which holds thread-local list of values to be pushed onto the queue at commit time. At commit time, the tailversion acts as a lock to prevent any other thread from pushing onto the queue. After locking tailversion, the thread pushes all elements on the \texttt{write\_list} onto the queue, and increments the tailversion.
If a transaction only performs pushes, then this transactions will always commit. This is because a push does not observe any property of the queue, such as the value at the head of the queue or the emptiness of the queue. 

A transactional pop first checks if the queue will be empty, given that the current transaction may intend to pop some values. If the queue will not be empty, the pop returns \texttt{true} and the thread must ensure at commit time that the head of the queue has not been modified by another thread. This is done by comparing the value of the headversion at the time of the pop with the value at commit time. 
If the queue will be empty, the thread checks if it should perform \emph{read-my-writes} by determining if it intends to push a value onto the queue in this transaction. If so, the thread removes the value from its \texttt{write\_list} and returns \texttt{true}. Otherwise, the return value of the pop is \texttt{false}. At commit time, the thread must check that the queue is still empty. This is done by checking the value of the tailversion, which increments each time an item is added to the queue.
When a transaction that performed pops commits, it locks the headversion (ensuring atomic access to the head of the queue), removes a value from the head of the queue for every successful transactional pop call, and increments the headversion.

%The design is summarized in Table \ref{table:sto1}.

\subsection{T-Queue2}
The T-Queue2 is also a circular, fixed-size queue, with operations push and pop. The T-Queue2 algorithm is a hybrid design integrating the T-Queue1 algorithm with another transactional algorithm: pessimistic locking. This takes inspiration from the transactional queue from the Transactional Data Structures Library\cite{tdsl} as described in previous work. Their pessimistic transactional queue appears to achieve better performance in their benchmarks than the T-Queue1, and the algorithm is simpler to implement and describe. 

Adding pessimistic locking is done by locking the queue when any pop (a naturally contentious operation) is invoked. The queue is then only unlocked after the transaction is complete. This ensures that no other thread will execute an operation that may invalidate a pop within this thread's transaction. However, operations such as ``push'' that can operate without any wait do not require locking during execution. Therefore, a push follows the same protocol as in the T-Queue1.

Because pop locks the queue, there are no conflicts at commit time. A thread only aborts if it fails to obtain the lock after a bounded period of time. The one version, “queueversion,” acts as the global queue lock. 

%%%%%%%%%%%%%%%%%%%%%%%%%%%%%%%%%%%%%%%%%%%%%%%%%%%%%%%%%%%%%%%%%%%%%%%%%%%%%%%%%%%%%%%%%%%%%%%%%%

\section{Flat Combining Queue Algorithms}
\lyt{evaluation?}Given the relatively slow performance of our STO queues, we looked to find a highly-concurrent (non-transactional) queue algorithm that would be promising to integrate with STO's transactional framework. After running several benchmarks (see Figures~\ref{fig:concurrent_qs} and~\ref{fig:concurrentqs_pushpop}), we found the most promising to be the Flat-Combining technique, which not only outperforms other queue algorithms, but also addresses several of the bottlenecks we observe in the STO queues.

\subsection{Non-Transactional Flat Combining Queue}
\label{fcqueuent}

Flat Combining, proposed by Hendler, et al. in 2010\cite{flatcombining}, is a synchronization technique that is based upon coarse-grained locking and single-thread access to the data structure. The key insight is that the cost of synchronization for certain classes of data structures often outweighs the benefits gained from parallelizing access to the data structure. These data structures include high-contention data structures such as stacks, queues, and priority queues. Created with this insight, the flat combining algorithm proposes a simple, thread-local synchronization technique that allows only one thread to ever access the data structure at once. This both reduces synchronization overhead on points of contention (such as the head of the queue) and achieves better cache performance by leveraging the single-threaded access patterns during data structure design.

The data structure design includes a sequential implementation of the data structure, a global lock, and per-thread records that are linked together in a global publication list. A record allows a thread to publish to other threads the specifics of any operation it wants to perform; the result of the operation is subsequently written to and retrieved from the record.

When a thread T wishes to do an operation O:
\begin{enumerate}
    \item T writes the opcode and parameters for O to its local record. Specifically for the queue, the thread writes \texttt{<PUSH, value>} or \texttt{<POP, 0>} to its local record.
   \item T tries to acquire the global lock. Depending on the result:
   \begin{enumerate}
        \item T acquires the lock and is now the “combiner” thread. T applies all thread requests in the publication list to the data structure in sequence, and writes both the result and an \texttt{<OK>} response to each requesting thread's local record.
        \item T failed to acquire the lock. T spins on its record until another thread has written the result to T's record with the response \texttt{<OK>}.
    \end{enumerate}
\end{enumerate}

In the context of the queue, flat combining proves to be an effective technique to handle the contention caused by parallel access on the head and tail of the queue. In addition, their choice of queue implementation uses ``fat nodes'' (arrays of values, with new nodes allocated when the array fills up), which both improves cache performance and allows the queue to be dynamically sized. Both the T-Queue1 and T-Queue2 suffer from the contention and cache performance issues pointed out in the flat combining paper, leading us to believe that flat combining's alternative synchronization paradigm might improve the performance of a transactional queue as much as it does for a concurrent queue.

\subsection{Transactional Flat Combining Queue} 
\label{fcqueuet}

Recall that, in addition to the requirements for a correct concurrent queue, a transaction queue must guarantee that there exists a serial order of all transactions such that, within one transaction, any two pops pop consecutive values in the queue starting from the head of the queue and any two pushes push consecutive values at the tail of the queue.

We must consider the order in which threads' requests are applied to the queue to be able to create a transactionally correct flat combining queue. For example, let a transaction in thread T1 be \texttt{\{pop, pop\}} and a transaction in thread T2 be \texttt{\{pop\}}. The combiner thread sees that T1 has published \texttt{<POP, 0>} and T2 has published \texttt{<POP, 0>} to the publication list. The combiner thread then applies \texttt{T1:Pop} (popping the head of the queue) and \texttt{T2:Pop} (popping the second item on the queue). When the next combining pass executes, the combiner thread will see that T1 has published \texttt{<POP, 0>} again to the queue. However, performing T1's second pop violates the queue's transactional specification: the two popped values in T1's transaction will not be consecutive. T1 must now abort, which means that T2's pop of the second-frontmost value in the queue is now invalid: it did not pop at the head of the queue.

Addressing the scenario described above requires two important changes to flat combining (we describe the rationale for these changes in Chapter~\ref{commutativity}): 
\begin{enumerate}
\item A push cannot be applied to the queue during a transaction's execution, and must instead be performed when a transaction commits.
\item An uncommitted pop in a thread's transaction must be unobservable by any other thread. This can be implemented in two ways:  
    \begin{enumerate}
        \item The algorithm can delay a transaction's pops until commit time. This then means the algorithm must track which values in the queue are going to be popped within the transaction. This prevents duplicate pops and detects if the queue will be ``empty'' by tracking how many values will be popped off the queue during this transaction. If another thread performs a pop or push during the transaction's lifetime, this can cause the transaction to abort since the ``empty'' status of the queue at commit time may now be inconsistent with what the transaction saw during execution. 
        \item The algorithm does not execute flat combining requests from other threads until the transaction has committed or aborted. Because only this thread can execute commands, pops can be performed at execution time (and restored to the head of the qeueue if the transaction aborts). This can be implemented either through making other threads' transactions abort, or by causing the other threads to block or spin.
    \end{enumerate}
\end{enumerate}

We now describe the new algorithms for push and pop.  We change the types of request a thread can publish to its record on the publication list. Recall that the original flat combining queue supports two requests: \texttt{<PUSH, value>} and \texttt{<POP, 0>}. The transactional queue supports the follow requests:
\begin{itemize}
    \item \texttt{<PUSH, list>} : push a list of values onto the queue
    \item \texttt{<MARK\_POP, thread\_id>} : mark a value in the queue to be popped by this \texttt{thread\_id}
    \item \texttt{<DEQ, thread\_id>} : dequeue all values in the queue that are marked by this \texttt{thread\_id}
    \item \texttt{<EMPTY?, thread\_id>} : check if the queue, after popping all items marked by this \texttt{thread\_id}, is empty
    \item \texttt{<CLEANUP, thread\_id>} : unmark all values that are marked with this \texttt{thread\_id}
\end{itemize}

As with the T-Queue1, a push within a transaction adds to an internal \texttt{write\_list\_item}. At commit time, the thread will invoke the \texttt{<PUSH, list>}, with the \texttt{write\_list} passed as the argument.

A pop is implemented with a pessimistic approach. Performing a pop within a transaction invokes the \texttt{<MARK\_POP, thread\_id>} command. The combiner thread, upon seeing a MARK\_POP command, looks at the first value at the head of the queue. If this value is marked with another thread's \texttt{thread\_id}, the combiner thread returns \texttt{<ABORT>} to the calling thread.

If the value is not marked, the combiner thread marks the value with the caller's \texttt{thread\_id} and returns \texttt{<OK>}. Note that in this scenario, no other thread can have marked values in the queue, since they will abort when seeing the head value marked by the calling thread's \texttt{thread\_id}. The combiner thread iterates sequentially through the queue values until it reaches a value not marked by the calling thread's \texttt{thread\_id}. It then marks the value with the caller's \texttt{thread\_id} and returns \texttt{<OK>}. Upon receiving the response, the calling thread adds a write to a \texttt{pop\_item} to tell the thread to post a \texttt{<DEQ, thread\_id>} request at commit time. This removes the popped value from the queue.

If the queue is either empty or there are no values not marked with the caller's \texttt{thread\_id}, the combiner thread will return \texttt{<EMPTY>}, which is remembered by the calling thread. An \texttt{<EMPTY>} response requires that the size of the queue be checked at commit time.

Note that this algorithm does not allow pops to read the values pushed within the same transaction. To do so would require passing in the thread's \texttt{write\_list} in addition to the \texttt{thread\_id} as arguments to the combiner thread. During our evaluation, we leave this part of the transactional queue specification unimplemented (and expect that adding this will only decrease performance).

The \texttt{<EMPTY?, thread\_id>} request is posted when a thread attempts to commit a transaction that observed an empty queue at some point in its execution. This happens when the thread receives an \texttt{<EMPTY>} response to a \texttt{<MARK\_POP>} request during the transaction. If the response to \texttt{<EMPTY?>} is true, then the thread knows that no other thread has pushed onto the queue between the time of its \texttt{<MARK\_POP>} seeing an empty queue and commit time. Else another thread has pushed items onto the queue, invalidating this thread's pop result, and this thread must abort.

If a thread ever sees an empty queue when executing a pop \emph{and} subsequently performs a push within the same transaction, the thread must prevent another transaction from committing between the time of the empty check and the installation of its pushed value. This requires adding what is essentially a lock of the tail of the queue. This is implemented via additional machinery in the combiner thread, which signals whether or not a transaction has locked the queue, and prevents any other thread's pushes from being installed until the ``lock'' is released.

The \texttt{<CLEANUP, thread\_id>} request is posted when a thread aborts a transaction and must unmark any items in the queue that it had marked as pending pops. The combiner thread iterates through the queue from the head and unmarks any items with the \texttt{thread\_id}.
