\chapter{Commutativity and Scalability in Transactional Queue Specifications}
\label{commutativity}

This chapter describes the commutativity of our queue operations both in a non-transactional setting and in a transactional setting, and relates the amount of queue operation commutativity to queue implementation performance. For clarity, we refer to the queue operation interface shown in Figure~\ref{fig:q_interface} as the \emph{strong queue specification}; a transactional queue with this interface is the \emph{strong transactional queue}. We hypothesize that the strong queue specification cannot be implemented in a transactional setting in an efficient way due to the lack of operation commutativity in the strong queue specification. 
We follow this by proposing an alternative queue specification---the \emph{weak queue specification}---that allows for greater operation commutativity, and hypothesize that this alternative specification will allow for greater queue scalability.

As a supporting example of our hypotheses, we examine the flat combining technique in detail and argue that the flat combining technique cannot implement the strong queue interface efficiently in a transactional setting. While the flat-combining technique is perhaps near-optimal for a highly-concurrent data structure, it performs no better than a naive synchronization technique in a transactional data structure. This is because the flat combining algorithm's high performance comes from exploiting the greater operation commutativity present in a non-transactional setting. The flat combining algorithm's optimizations must be heavily modified in order to support transactions, which leads to significant performance loss. 

We then implement a weak transactional flat combining queue---a flat combining queue with operations satisfying the weak queue specification---with the expectation that the flat combining technique can achieve scalable performance close to its performance in a non-transactional setting. Our experimental results illustrate that the greater commutativity of operations in the weak queue specification is essential for the flat combining technique to be effective.

\section{History Terminology}

We introduce some terminology about histories and transactional histories that will be used in our discussion of operation commutativity.

\begin{defn}
    A \emph{history} is a sequence of \texttt{(thread, operation, result)} tuples that represent an interleaving of operations of all threads. Knowledge of both the history and initial conditions of a data structure leads to complete knowledge of the (high-level) end state of the structure.
\end{defn}

\begin{eg}
    \singlespacing   

    \begin{lstlisting}

    // Q.size() == 0 
    (T2, Q.push(a), ())
    (T1, Q.pop(), true)
    (T2, Q.push(a), ())
    (T1, Q.pop(), true)
    // Final State: Q.size() == 0 
    \end{lstlisting}
    \doublespacing
\end{eg}
\begin{defn}
    A \emph{transactional history} is a specific type of history in which the tuples represent an interleaving of operations of the threads' committed transactions. A transactional history includes \texttt{(thread, START\_TXN, ())} and \texttt{(thread, COMMIT\_TXN, commit\_result)} operation tuples that represent the time the thread starts and commits the transaction. \texttt{commit\_result} represents the observable effects of the installation procedure at commit time.

\begin{eg}
    \singlespacing   

    \begin{lstlisting}

    // Q.size() == 0 
    (T1, START_TXN, ())
    (T2, START_TXN, ())
    (T2, Q.push(a), ())
    (T1, Q.pop(), true)
    (T2, Q.push(a), ())
    (T1, Q.pop(), true)
    (T1, COMMIT_TXN, ())
    (T2, COMMIT_TXN, ())
    // Final State: Q.size() == 0 
    \end{lstlisting}
    \doublespacing
\end{eg}

\end{defn}

\begin{defn}
    A history $H'$ is \emph{consistent} with $H$ if:
    \begin{enumerate}
        \item $H'$ contains the same tuples as $H$: the same operations were executed with the same return values for all operations within the transactions.
        \item The order of a single thread's calls in $H'$ remains consistent with the thread's order of calls in $H$.
    \end{enumerate}
\end{defn}

\begin{defn}
    A transactional history $H$ is \emph{serial} if all tuples are grouped by transaction: if $i\le j\le k$ and $H_i$ and $H_k$ are from the same transaction, then $H_j$ is also from that transaction. This means the tuples form a serial transaction order.
\end{defn}
\begin{defn}
    A transactional history $H$ is \emph{serializable} if there exists a serial history $H'$ s.t. $H'$ is consistent with $H$.

\end{defn}

\begin{eg}
    $H$ is a serializable transactional history whose corresponding serial execution is $H'$. $H''$ represents a serial transactional history, but is inconsistent with $H$ because its pop operations return different results.
\begin{figure}[H]
\singlespacing   
   \begin{tabular}{c|c|c}
H & H' & H''\\
\hline
\begin{lstlisting}
// Q.size() == 0 
(T1, START_TXN, ())
(T2, START_TXN, ())
(T2, Q.push(a), ())
(T1, Q.pop(), true)
(T2, Q.push(a), ())
(T1, Q.pop(), true)
(T1, COMMIT_TXN, ())
(T2, COMMIT_TXN, ())
\end{lstlisting} & 
\begin{lstlisting}
// Q.size() == 0 
(T2, START_TXN)
(T2, Q.push(a), ())
(T2, Q.push(a), ())
(T2, COMMIT_TXN)
(T1, START_TXN)
(T1, Q.pop(), true)
(T1, Q.pop(), true)
(T1, COMMIT_TXN)
\end{lstlisting} &
\begin{lstlisting}
// Q.size() == 0 
(T1, START)
(T1, Q.pop(), false)
(T1, Q.pop(), false)
(T1, COMMIT_TXN)
(T2, START)
(T2, Q.push(a), ())
(T2, Q.push(a), ()) 
(T2, COMMIT_TXN)
\end{lstlisting}
\end{tabular}
\end{figure}
\end{eg}

\begin{defn}
A transactional history is \emph{linearizable} if all transactions appears to occur instantaneously between their start time and their commit time: if transaction $T1$ commits before transaction $T2$, then $T1$ must appear before $T2$ in the serial history~\cite{harristm}.
\end{defn}

\begin{defn}
    A transactional history $H$ is \emph{strictly serializable}, or \emph{valid}, if it is both serializable and linearizable. Any data structure implemented in a transactional setting requires strictly serializable transactional histories.
\end{defn}

\begin{eg}
$H$ is a serializable, but not linearizable transactional history. This is because $T2$ should have observed the pushes committed by $T1$. We can find a serial ordering of $H$, shown in $H'$, but $H'$ violates the rule that the serial order of transactions corresponds to the real time order of the transactions' commits.
    
\begin{figure}[H]
    \centering
\singlespacing   
    \begin{tabular}{c|c}
H & H'\\
\hline
\begin{lstlisting}
// Q empty                          
(T1, START_TXN)
(T1, Q.push(a), ())                
(T1, Q.push(a), ())               
(T1, Q.pop(), true)
(T1, COMMIT_TXN)
(T2, START_TXN)
(T2, Q.pop(), false)
(T2, COMMIT_TXN)
\end{lstlisting} & 
\begin{lstlisting}
// Q empty
(T2, START_TXN)
(T2, Q.pop(), false)
(T2, COMMIT_TXN)
(T1, START_TXN)
(T1, Q.push(a), ())                       
(T1, Q.push(a), ())
(T1, Q.pop(), true)
(T1, COMMIT_TXN)
\end{lstlisting}
    \end{tabular}
\end{figure}
\end{eg}

\section{The Relationship of Commutativity and Scalability}

The \emph{scalable commutativity rule}, formally defined by Clements et al.~\cite{scrule}, asserts that whenever interface operations \emph{commute}, there exists an implementation of the interface that scales.
Operations \emph{commute} in a particular interface when there is no way to distinguish their execution order: exchanging the order of the operations in the history does not modify the return values of the operations seen by each thread.

\subsection{Commutativity of the Strong Queue Specification} 

In a non-transactional setting, we consider histories in which the only operations are push and pop (i.e., the histories are not transactional). Given the strong queue specification, in which push returns \texttt{void} and pop returns \texttt{bool}, we determine the commutativity of these operations by examining if exchanging the order in which the operations appear in the history changes the operations' return values. We show operations that do not commute in Table~\ref{tab:strongq_commute}.

Based on this commutativity analysis, we note that a pop operation does not commute with a push operation when the queue is empty, or another pop operation when the queue is near empty. By the scalable commutativity rule, this means that there is no concurrent queue implementation for pop that scales in these particular scenarios. A push operation commutes with all operations because it returns \texttt{void}, and has a scalable implementation in all scenarios.

\begin{table}[t]
    \singlespace
    \centering
    \begin{tabular}{|c|l|l|}
        \hline
        Operations & \multicolumn{1}{|c|}{H} H & \multicolumn{1}{|c|}{H'} H' \\
        \hline
    
        push and pop on an empty queue &
    \begin{lstlisting}
    // Q empty
    (T, Q.push(a), ())                       
    (T, Q.pop(), true)
    \end{lstlisting} &
    \begin{lstlisting}
    // Q empty
    (T, Q.pop(), false)
    (T, Q.push(a), ())                       
    \end{lstlisting}\\
    \hline
    
        pop and pop on a queue with 1 value &
    \begin{lstlisting}
    // Q.size() = 1
    (T1, Q.pop(), true)
    (T2, Q.pop(), false)                       
    \end{lstlisting} &
    \begin{lstlisting}
    // Q.size() = 1
    (T2, Q.pop(), true)
    (T1, Q.pop(), false)                       
    \end{lstlisting}\\
    \hline

    \end{tabular}
    \caption{Strong queue operations that do not commute}
    \label{tab:strongq_commute}
    \end{table}

We reason about commutativity of a queue implemented in a transactional setting using transactional histories, which include \texttt{START\_TXN} and \texttt{COMMIT\_TXN} operations. A transactional setting calls for strict serializability of the transactional history, which by definition entails serializability (i.e., that tuples in histories be grouped by transaction). This adds an additional level of commutativity, namely commutativity between transactions.

Because a valid transactional history is strictly serializable, we can find a corresponding serial history for every valid transactional history. This means that operations belonging to the same transaction must occur in a group in the history. To reason about transaction commutativity, we have a parallel notion to exchanging operations in the history to detect whether two operations commutate:  we exchange \emph{groups} of operations in a \texttt{START\_TXN} and \texttt{COMMIT\_TXN} block of a serial history, and detect whether there is any observable change in the return values within the exchanged transactions. Because the occurrence of each transaction in the serial transactional history can be uniquely identified by its \texttt{COMMIT\_TXN} tuple, if exchanging the positions of two transactions does not commute in a particular scenario, we can say that the \texttt{COMMIT\_TXN} operation does not commute in that scenario.

We show scenarios in which \texttt{COMMIT\_TXN} operations do not commute in Table~\ref{tab:txnal_strongq_commute}. Note that when transactions contain only one operation, then a lack of commutativity between two transactions is equivalent to saying that that the two operations do not commute. Since we already discussed individual operation commutativity (Table~\ref{tab:strongq_commute}), we omit those scenarios here. We also omit redundant scenarios: for example, let transactions $T1$ perform operations $O1$ and transactions $T2$ perform operations $O2$. Assume $T1$ and $T2$ do not commute: the return values of operations $O1$ change when the order of $T1$ and $T2$ is exchanged in the history. Then it is clear that a transaction performing $O1 + O3$ does not commute with $T2$, for some arbitrary operations $O3$.

\begin{table}[t]
    \singlespace
    \centering
    \begin{tabular}{|c|l|l|}
        \hline
        Operations & \multicolumn{1}{|c|}{H} H & \multicolumn{1}{|c|}{H'} H' \\
        \hline
   
        pop-push and pop on an empty queue
    \begin{lstlisting}
    // Q empty
    (T1, START_TXN, ())                       
    (T1, Q.pop(), false)                       
    (T1, Q.push(a), ())                       
    (T1, COMMIT_TXN, ())                       
    (T2, START_TXN, ())                       
    (T2, Q.pop(), true)                       
    (T2, COMMIT_TXN, ())                       
    \end{lstlisting} &
    \begin{lstlisting}
    // Q empty
    (T2, START_TXN, ())                       
    (T2, Q.pop(), false)                       
    (T2, COMMIT_TXN, ())                       
    (T1, START_TXN, ())                       
    (T1, Q.pop(), false)                       
    (T1, Q.push(a), ())                       
    (T1, COMMIT_TXN, ())                       
    \end{lstlisting}\\
    \hline
     
        pop-push and push
    \begin{lstlisting}
    // Q empty
    (T1, START_TXN, ())                       
    (T1, Q.pop(), false)                       
    (T1, Q.push(a), ())                       
    (T1, COMMIT_TXN, ())                       
    (T2, START_TXN, ())                       
    (T1, Q.push(a), ())                       
    (T2, COMMIT_TXN, ())                       
    \end{lstlisting} &
    \begin{lstlisting}
    // Q empty
    (T2, START_TXN, ())                       
    (T1, Q.push(a), ())                       
    (T2, COMMIT_TXN, ())                       
    (T1, START_TXN, ())                       
    (T1, Q.pop(), true)                       
    (T1, Q.push(a), ())                       
    (T1, COMMIT_TXN, ())                       
    \end{lstlisting}\\
    \hline
    
    \end{tabular}
    \caption[Strong queue transactions that do not commute]{Strong queue transactions that do not commute. For clarity, we show only the serial history corresponding to the original valid transactional history.}
    \label{tab:txnal_strongq_commute}
    \end{table}
We hypothesize that any implementation of our queue interface that satisfies the strongly-transactional specification will not scale. This is because a pop operation in one transaction never commutes with any operation performed by another transaction when encountering an empty queue, and pops performed in two separate transactions never commute. Because of this lack of operation commutativity, the transactional queue specification has a large number of invalid dependency cycles that must be handled; this prevents efficient implementation of a transactional queue with our interface. To demonstrate how our queue interface and the corresponding commutativity of pop operations prevent a scalable strongly-transactional queue implementation, imagine a different queue interface in which a pop operation returns \texttt{void} instead of \texttt{bool}. A transaction executing a pop would not observe the empty status of the queue, which would allow a pop to commute with all other operations in the history (exchanging the order of a pop and another operation would have no observable effect to the caller). With this interface, a queue satisfying the strongly-transactional specification could simply execute all operations at commit time using any concurrent queue algorithm, and do nothing (except return \texttt{void}) at execution time. This would make the queue equally as scalable as any concurrent queue algorithm.

\subsection{Commutativity and Flat Combining Performance}
An example of a queue algorithm that cannot effectively implement a strongly-transactional queue with our queue interface is the flat combining algorithm. 
Flat combining's fundamental principle is that all requests posted to the publication list commute with each other: the combiner thread can blindly apply operations on the publication list in an arbitrary order. In a strongly-transactional queue with our queue interface, push and pop operations will no longer always commute.
This requires additional complexity of each flat combining call (pop and push) and additional flat combining calls to check, undo, or install at commit time. We describe several methods to do so below, and argue that these methods cannot be integrated with the flat combining algorithm without introducing overhead that reduces its performance to below that of the T-QueueO or T-QueueP.

There are several methods for preventing cyclic dependencies from forming (i.e., preventing invalid operation interleavings from occurring in the history). For example, a standard method for a transactional queue is to delay push operation execution, and to use a pessimistic or optimistic approach when encountering an empty queue. This method can prevent all invalid interleavings shown in Table~\ref{tab:interleavings}.

To prevent interleavings 4 and 5, all pushes are delayed until commit time. These interleavings can occur only if $T1$'s first push is visible to $T2$ prior to $T1$'s commit. If we delay pushes until commit time, $T2$ will not detect the presence of a pushed item in the queue.

Because pop operations immediately return values that depend on the state of the queue (\texttt{false} if the queue is empty or \texttt{true} if the queue is nonempty), interleavings 1, 2, and 3 cannot be prevented by delaying pop operations until commit time. Instead, we can take one of two approaches. Let $T1$ be a transaction that has performed a pop.
\begin{enumerate}
    \item Optimistic: Abort $T1$ at commit time if $T2$ has committed an operation that would cause an invalid interleaving.
    \item Pessimistic: Prevent $T2$ from committing any operation until after $T1$ commits or aborts.
\end{enumerate}

The T-QueueO implements the optimistic method: checks of the tail version and the head version determine at commit time whether the empty status of the queue has been modified by another, already committed transaction. The T-QueueP implements the pessimistic approach, which locks the queue after a pop is performed and only releases the lock if the transaction commits or aborts, therefore preventing any other transaction from committing any operation after the pop.

The flat combining approach can do either approach (1) or (2) to support transactions; however, the flat combining approach cannot do either without introducing overhead that reduces its performance to below that of the T-QueueO or T-QueueP.

If we take approach (1), a pop cannot be performed at execution time because no locks on the queue are acquired at execution time: other transactions are allowed to commit pops, which may pop an invalid head if this transaction aborts. Thus, in order to determine if a pop should return true or return false, a transactional pop flat combining request requires much more complexity than a non-transactional one: the thread must determine how many elements the queue holds, how many elements the current transaction is intending to pop, and if any other thread intends to pop (in which case the transaction aborts). The transactional push flat combining request is also more complex, as it requires installing all the pushes of the transaction. Additional flat combining calls are necessary to allow a thread to perform checks of the queue's empty status (the \texttt{<EMPTY?>} flat combining call) to determine whether the transaction can commit or must abort, and to actually execute the pops at commit time. Thus, approach (1) requires adding both more flat combining calls and more complexity to the existing flat combining calls.

If we take approach (2), the flat combining approach can either perform a pop at execution time or delay the pop until commit time. If the pop is performed at execution time, then the thread must acquire a global lock on the queue after a pop and hold the lock until commit: this prevents another thread from observing an inconsistent state of the queue. If a pop removes the head of the queue prior to commit and the transaction later aborts, the popped element must be re-attached to the head of the queue. Any thread performing a pop must acquire a global lock to ensure that no other thread can commit a transaction that pops off the incorrect head of the queue (given that elements may be reattached to the head if the transaction aborts). Additional flat combining calls are necessary to acquire or release the global lock. 

We can also imagine a mix of approaches (1) and (2). If a transaction $T1$ executes a pop, we can disallow any pops from other transactions (using the equivalent of a global lock) but allow other transactions containing only pushes to commit prior to $T1$ completing. This approach prevents interleavings 1 and 3, but requires performing a check of the queue's empty status, as in approach (1), if a pop saw the queue in an empty state. This is because another transaction may have committed a push between the time of $T1$'s pop and $T1$'s completion. This mixed approach outperforms both approach (2) and approach (1), and is the approach described as the flat combining algorithm in Chapter~\ref{queue}. 

As previously noted, all possible approaches to prevent interleavings 1, 2, and 3 rely on implementing additional flat combining calls and increasing the complexity of previously existing flat combining calls. In addition, acquisition of a global ``lock'' on the queue for approach (2) prevents the combiner thread from applying \emph{all} of the requests it sees; instead, requests will either return ``abort'' to the calling thread or not be applied, leading to additional time spent spinning or repeating requests. These modifications to the flat combining algorithm allow the combiner thread to prevent the interleavings that lead to invalid dependency cycles in the strongly-transactional specification.

We see through our experiments that these changes to the flat combining algorithm reduce its performance such that it performs worse than a naive synchronization algorithm; furthermore, we claim that these changes, or changes similar in nature, are necessary in order to provide transactional guarantees. The original flat combining algorithm exploits the property that any correct history of operations in data structures supporting only singleton transactions (i.e., a concurrent, non-transactional data structure) is valid. The combiner thread is allowed to immediately apply all threads' operation requests in arbitrary order. However, this property that makes flat combining so performant disappears as soon as the algorithm has to deal with non-serializable histories. In the next section, we demonstrate how changing the transactional specification to allow for greater operation commutativity leads to a version of flat combining that can outperform our T-QueueO and T-QueueP algorithms: this supports our claim that the flat combining algorithm's performance is heavily dependent on what types of transactional guarantees it must provide.
