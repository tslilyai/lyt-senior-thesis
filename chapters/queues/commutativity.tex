\chapter{Commutativity and Scalability in Transactional Queue Specifications}
\label{commutativity}

This chapter describes the commutativity of our queue operations both in a non-transactional setting and in a transactional setting, and relates the amount of queue operation commutativity to queue implementation performance. For clarity, we refer to the queue operation interface shown in Figure~\ref{fig:q_interface} as the \emph{strong queue specification}; a transactional queue with this interface is the \emph{strong transactional queue}. We hypothesize that the strong queue specification cannot be implemented in a transactional setting in an efficient way due to the lack of operation commutativity in the strong queue specification. 
We follow this by proposing an alternative queue specification---the \emph{weak queue specification}---that allows for greater operation commutativity, and hypothesize that this alternative specification will allow for greater transactional queue scalability.

As a supporting example of our hypotheses, we examine the flat combining technique in detail and argue that the flat combining technique cannot implement the strong queue interface efficiently in a transactional setting. While the flat-combining technique is perhaps near-optimal for a highly-concurrent data structure, it performs no better than a naive synchronization technique in a transactional data structure. This is because the flat combining algorithm's high performance comes from exploiting the greater operation commutativity present in a non-transactional setting. The flat combining algorithm's optimizations must be heavily modified in order to support transactions, which leads to significant performance loss. 

We then implement a weak transactional flat combining queue---a flat combining queue with operations satisfying the weak queue specification---with the expectation that the flat combining technique can achieve scalable performance close to its performance in a non-transactional setting. Our experimental results illustrate that the greater commutativity of operations in the weak queue specification is essential for the flat combining technique to be effective in a transactional setting.

\section{History Terminology}

We introduce some terminology about histories and transactional histories that will be used in our discussion of operation commutativity.

\begin{defn}
    A \emph{history} is a sequence of \texttt{(thread, operation, result)} tuples that represent an interleaving of operations of all threads. Knowledge of both the history and initial conditions of a data structure leads to complete knowledge of the (high-level) end state of the structure.
\end{defn}

\begin{eg}
    \singlespacing   

    \begin{lstlisting}

    // Q.size() == 0 
    (T2, Q.push(a), ())
    (T1, Q.pop(), true)
    (T2, Q.push(a), ())
    (T1, Q.pop(), true)
    // Final State: Q.size() == 0 
    \end{lstlisting}
    \doublespacing
\end{eg}
\begin{defn}
    A \emph{transactional history} is a specific type of history in which the tuples represent an interleaving of operations of the threads' committed transactions. A transactional history includes \texttt{(thread, START\_TXN, ())} and \texttt{(thread, COMMIT\_TXN, commit\_result)} operation tuples that represent the time the thread starts and commits the transaction. \texttt{commit\_result} represents the observable effects of the installation procedure at commit time.

\begin{eg}
    \singlespacing   

    \begin{lstlisting}

    // Q.size() == 0 
    (T1, START_TXN, ())
    (T2, START_TXN, ())
    (T2, Q.push(a), ())
    (T1, Q.pop(), true)
    (T2, Q.push(a), ())
    (T1, Q.pop(), true)
    (T1, COMMIT_TXN, ())
    (T2, COMMIT_TXN, ())
    // Final State: Q.size() == 0 
    \end{lstlisting}
    \doublespacing
\end{eg}

\end{defn}

\begin{defn}
    A history $H'$ is \emph{consistent} with $H$ if:
    \begin{enumerate}
        \item $H'$ contains the same tuples as $H$: the same operations were executed with the same return values for all operations within the transactions.
        \item The order of a single thread's calls in $H'$ remains consistent with the thread's order of calls in $H$.
    \end{enumerate}
\end{defn}

\begin{defn}
    A transactional history $H$ is \emph{serial} if all tuples are grouped by transaction: if $i\le j\le k$ and $H_i$ and $H_k$ are from the same transaction, then $H_j$ is also from that transaction. This means the tuples form a serial transaction order.
\end{defn}
\begin{defn}
    A transactional history $H$ is \emph{serializable} if there exists a serial history $H'$ s.t. $H'$ is consistent with $H$.

\end{defn}

\begin{eg}
    $H$ is a serializable transactional history whose corresponding serial execution is $H'$. $H''$ represents a serial transactional history, but is inconsistent with $H$ because its pop operations return different results.
\begin{figure}[H]
\singlespacing   
   \begin{tabular}{c|c|c}
H & H' & H''\\
\hline
\begin{lstlisting}
// Q.size() == 0 
(T1, START_TXN, ())
(T2, START_TXN, ())
(T2, Q.push(a), ())
(T1, Q.pop(), true)
(T2, Q.push(a), ())
(T1, Q.pop(), true)
(T1, COMMIT_TXN, ())
(T2, COMMIT_TXN, ())
\end{lstlisting} & 
\begin{lstlisting}
// Q.size() == 0 
(T2, START_TXN)
(T2, Q.push(a), ())
(T2, Q.push(a), ())
(T2, COMMIT_TXN)
(T1, START_TXN)
(T1, Q.pop(), true)
(T1, Q.pop(), true)
(T1, COMMIT_TXN)
\end{lstlisting} &
\begin{lstlisting}
// Q.size() == 0 
(T1, START_TXN)
(T1, Q.pop(), false)
(T1, Q.pop(), false)
(T1, COMMIT_TXN)
(T2, START_TXN)
(T2, Q.push(a), ())
(T2, Q.push(a), ()) 
(T2, COMMIT_TXN)
\end{lstlisting}
\end{tabular}
\end{figure}
\end{eg}

\begin{defn}
A transactional history is \emph{linearizable} if all transactions appears to occur instantaneously between their start time and their commit time: if transaction $T1$ commits before transaction $T2$, then $T1$ must appear before $T2$ in the serial history~\cite{harristm}.
\end{defn}

\begin{defn}
    A transactional history $H$ is \emph{strictly serializable}, or \emph{valid}, if it is both serializable and linearizable. Any data structure implemented in a transactional setting requires strictly serializable transactional histories.
\end{defn}

\begin{eg}
$H$ is a serializable, but not linearizable transactional history. This is because $T2$ should have observed the pushes committed by $T1$. We can find a serial ordering of $H$, shown in $H'$, but $H'$ violates the rule that the serial order of transactions corresponds to the real time order of the transactions' commits.
    
\begin{figure}[H]
    \centering
\singlespacing   
    \begin{tabular}{c|c}
H & H'\\
\hline
\begin{lstlisting}
// Q empty                          
(T1, START_TXN)
(T1, Q.push(a), ())                
(T1, Q.push(a), ())               
(T1, Q.pop(), true)
(T1, COMMIT_TXN)
(T2, START_TXN)
(T2, Q.pop(), false)
(T2, COMMIT_TXN)
\end{lstlisting} & 
\begin{lstlisting}
// Q empty
(T2, START_TXN)
(T2, Q.pop(), false)
(T2, COMMIT_TXN)
(T1, START_TXN)
(T1, Q.push(a), ())                       
(T1, Q.push(a), ())
(T1, Q.pop(), true)
(T1, COMMIT_TXN)
\end{lstlisting}
    \end{tabular}
\end{figure}
\end{eg}

\section{The Relationship of Commutativity and Scalability}

The \emph{scalable commutativity rule}, formally defined by Clements et al.~\cite{scrule}, asserts that whenever interface operations \emph{commute}, there exists an implementation of the interface that scales.
Operations \emph{commute} in a particular interface when there is no way to distinguish their execution order: exchanging the order of the operations in the history does not modify the return values of the operations seen by each thread.

\subsection{Commutativity of the Strong Queue Specification} 

In a non-transactional setting, we consider histories in which the only operations are push and pop (i.e., the histories are not transactional). Given the strong queue specification (Figure~\ref{fig:q_interface}) in which push returns \texttt{void} and pop returns \texttt{bool}, we determine the commutativity of these operations by examining if exchanging the order in which the operations appear in the history changes the operations' return values. We show operations that do not commute in Table~\ref{tab:strongq_commute}.
Based on this commutativity analysis, we note that a pop operation does not commute with a push operation when the queue is empty, or another pop operation when the queue is near empty. By the scalable commutativity rule, this means that there is no concurrent queue implementation for pop that scales in these particular scenarios. A push operation commutes with all operations because it returns \texttt{void}, and has a scalable implementation in all scenarios.

\begin{table}[t]
    \singlespace
    \centering
    \begin{tabular}{|c|l|l|}
        \hline
        Operations & \multicolumn{1}{c}{H} & \multicolumn{1}{|c|}{H'} \\
        \hline
    
        push vs. pop &
\begin{lstlisting}
// Q empty
(T, Q.push(a), ())                       
(T, Q.pop(), true)
\end{lstlisting} &
\begin{lstlisting}
// Q empty
(T, Q.pop(), false)
(T, Q.push(a), ())                       
\end{lstlisting}\\
\hline

    pop vs. pop &
\begin{lstlisting}
// Q.size() = 1
(T1, Q.pop(), true)
(T2, Q.pop(), false)                       
\end{lstlisting} &
\begin{lstlisting}
// Q.size() = 1
(T2, Q.pop(), true)
(T1, Q.pop(), false)                       
\end{lstlisting}\\
    \hline

    \end{tabular}
    \caption[Strong queue operations that do not commute]{Strong queue operations that do not commute. H is the original history; H' is the history with the order of the two operations exchanged. In the push vs pop scenario, note that any thread may have performed the operations, and the operations will still not commute.}
    \label{tab:strongq_commute}
    \end{table}

We reason about commutativity of a queue implemented in a transactional setting using transactional histories, which include \texttt{START\_TXN} and \texttt{COMMIT\_TXN} operations. A transactional setting calls for strict serializability of the transactional history, which by definition entails serializability (i.e., that tuples in histories be grouped by transaction). This adds an additional level of commutativity, namely commutativity between transactions.

Because a valid transactional history is strictly serializable, we can find a corresponding serial history for every valid transactional history. This means that operations belonging to the same transaction must occur in a group in the history. To reason about transaction commutativity, we have a parallel notion to exchanging operations in the history to detect whether two operations commutate:  we exchange \emph{groups} of operations in a \texttt{START\_TXN} and \texttt{COMMIT\_TXN} block of a serial history, and detect whether there is any observable change in the return values within the exchanged transactions. Because the occurrence of each transaction in the serial transactional history can be uniquely identified by its \texttt{COMMIT\_TXN} tuple: if exchanging the positions of two transactions does not commute in a particular scenario, we can say that the two \texttt{COMMIT\_TXN} operations do not commute in that scenario.

We also must now consider when \texttt{COMMIT\_TXN} operations do not commute. Note that when transactions contain only one operation, then a lack of commutativity between two transactions is equivalent to saying that that the two operations do not commute (and vice versa). We provide a constructive approach to find all non-singleton transactions in the history whose lack of commutativity prevents their order from being exchanged. This approach works by extending the singleton operations transactions that do not commute---identified in Table~\ref{tab:strongq_commute}---as follows: 
\begin{itemize}
    \item Let transaction $T1$ perform some ordered set of operations $O_1$ and transaction $T2$ perform some ordered set of operations $O_2$. Let $H$ be a (serial) history in which $T1$ and $T2$ do not commute, and $H'$ be the history in which the order of $T1$ and $T2$ has been exchanged. Let $H_{new}$ and $H'_{new}$ be the histories when $T1$ or $T2$ is extended with new operations.
    \item Assume $T1$ and $T2$ do not commute: WLOG, let at least one of the return values of an operation in $O_1$ change when the order of $T1$ and $T2$ is exchanged in $H$. We know this happens only if exchanging the order of $T1$ and $T2$ causes a pop operation in $T1$ to observe an empty queue given one order, but not in another.
    \item We extend $T1$ with operations $O_3$ appended to $O_1$, where $O_3$ is some arbitrary set of operations. Then $T1$ performing $O_1 + O_3$ does not commute with $T2$. This is because we will have in $O_1 + O_3 + O_2$ in $H_{new}$, and $O_2 + O_1 + O_3$ in $H'_{new}$: the return values of $O_1$ in $O_1 + O_3 + O_2$ are the same as in $O_1 + O_2$, and the return values of $O_1$ in $O_2 + O_1 + O_3$ are the same as in $O_2 + O_1$.
    \item We extend $T1$ with operations $O_4$ added to the front of $O_1$. $O_4$ is an ordered set of operations such that at least one pop in $O_1$ observes an empty queue in $H_{new}$ or $H'_{new}$, and a nonempty queue in the other. Then $T1$ performing $O_4 + O_1$ does not commute with $T2$ by definition.
\end{itemize}

Conversely, we know that these arbitrarily large, non-commuting transactions that we find through our construction are the \emph{only} transactions that do not commute: 
\begin{itemize}
    \item Let transaction $T1$ perform some ordered set of operations $P$ and transaction $T2$ perform some ordered set of operations $Q$.
    \item We are given that $T1$ and $T2$ do not commute. WLOG, this means that we can find some $o \in P$ s.t. $o$'s return value changes when we exchange the order of $T'$ and $T''$ in the history, and $o$ is the first such operation (necessarily a pop) in $P$ s.t.\ this occurs.
    \item We deconstruct $O$ into the following ordered set: $O' + \{o\} + O''$. Let $\{o\}$ be $O_1$ from the construction above. 
        Let $O'$ be $O_4$: $O'$ satisfies the properties of $O_4$ specified above, because $o$ provides the witness for the pop in $O_1$ that observes an empty queue $H_{new}$ or $H'_{new}$ and a nonempty queue in the other. We can also find a witness for $O_3$, namely $O''$: because $O' + \{o\}$ does not commute with $T1$, our construction will generate $P = O' + \{o\} + O''$ as operations of a transaction $T1$ that does not commute with $T2$.
\end{itemize}

\begin{table}[t]
    \singlespace
    \centering
    \begin{tabular}{|c|l|l|}
        \hline
        Example & \multicolumn{1}{c}{H} & \multicolumn{1}{|c|}{H'} \\
        \hline
        1. & 
\begin{lstlisting}
// Q empty
(T1, START_TXN, ())                       
(T1, Q.pop(), false)                       
(T1, Q.push(a), ())                       
(T1, COMMIT_TXN, ())                       
(T2, START_TXN, ())                       
(T2, Q.pop(), true)                       
(T2, COMMIT_TXN, ())                       
\end{lstlisting} &
\begin{lstlisting}
// Q empty
(T2, START_TXN, ())                       
(T2, Q.pop(), false)                       
(T2, COMMIT_TXN, ())                       
(T1, START_TXN, ())                       
(T1, Q.pop(), false)                       
(T1, Q.push(a), ())                       
(T1, COMMIT_TXN, ())                       
\end{lstlisting}\\
\hline
    2. &
\begin{lstlisting}
// Q empty
(T1, START_TXN, ())                       
(T1, Q.push(a), ())                       
(T1, Q.pop(), true)                       
(T1, Q.push(a), ())                       
(T1, COMMIT_TXN, ())                       
(T2, START_TXN, ())                       
(T2, Q.pop(), (true))                       
(T2, Q.push(a), ())                       
(T2, COMMIT_TXN, ())                       
\end{lstlisting} &
\begin{lstlisting}
// Q empty
(T2, START_TXN, ())                       
(T2, Q.pop(), (false))                       
(T2, Q.push(a), ())                       
(T2, COMMIT_TXN, ())                       
(T1, START_TXN, ())                       
(T1, Q.push(a), ())                       
(T1, Q.pop(), true)                       
(T1, Q.push(a), ())                       
(T1, COMMIT_TXN, ())                       
\end{lstlisting}\\
    \hline
    
    \end{tabular}
    \caption[Examples of strong queue transactions that do not commute]{Examples of strong queue transactions that do not commute. For clarity, we show only the serial history corresponding to the original valid transactional history.
    H is the original serial history; H' is the history with the order of the two transactions exchanged.}
    \label{tab:txnal_strongq_commute}
    \end{table}

We provide some examples of larger transactions that fail to commute in Table~\ref{tab:strongq_commute}.

Based on this commutativity analysis, we note that, in addition to the non-commutativity of pop operations in particular scenarios involving empty, or near-empty queues, we now have a further lack of commutativity of transactions (identified by their \texttt{COMMIT\_TXN} operations) even in scenarios in which the queue may contain far more than one element. For example, even if a transaction starts with a relatively non-empty queue, a large transaction that performs several pops may reduce the queue to a near-empty state, and therefore contain a pop operation that observes the empty status of the queue. This transaction will likely \emph{not} commute with any other transaction performing a push or pop.  By the scalable commutativity rule, this means that there is no scalable queue implementation for a strong transactional queue whenever \texttt{COMMIT\_TXN} operations do not commute.

We hypothesize that any implementation of our strong queue specification in a transactional setting that must handle scenarios in which queues may become empty will not scale. A concurrent queue already lacks a scalable implementation for pop due to the high likelihood that a pop operation will not commute with any other operation when the queue nears empty. When this queue is put in a transactional setting, there is an even higher likelihood that two transactions will not commute, even if the queue contains more values. This prevents an efficient, scalable implementation of a strong transactional queue. In Section~\ref{wqueue}, we provide a counterexample, a transactional queue satisfying the \emph{weak queue specification}, and demonstrate how increased commutativity in this new specification allows for a scalable implementation.

\subsection{Strong Transactional Flat Combining Queue Performance}
The flat combining algorithm is an example of a queue algorithm that implements the strong queue interface and loses its effectiveness in a transactional setting. Recall that our results from testing Hypothesis 5 demonstrate that flat combining's effectiveness is lost in a transactional setting. Here, we argue that this is due to a decrease in the amount of commutativity in the transactional setting: in addition to non-commutative single operations (pops and pushes), transactions may not always commute.

Flat combining's fundamental principle is that requests posted to the publication list can be blindly applied to the queue in an arbitrary order. It handles the non-commutativity on the level of individual pop operations by synchronizing concurrent access to the queue with the combiner thread: only one thread is allowed to perform an operation on the queue at a time.\footnote{We note that flat combining is not a scalable implementation of pop (or push): the combiner thread's application of requests can only be as efficient as a sequential execution of all thread operations. As our results from testing Hypothesis 3 show, flat combining is nonetheless the most efficient of all concurrent queue algorithms evaluated.}
In a strong transactional queue, the flat combining algorithm must also correctly handle transactions that may not commute, which means synchronizing the \texttt{COMMIT\_TXN} operations of different transactions. This requires that the algorithm prevent operations within transactions from interleaving in the history in a way that prevents the history from being serialized. In other words, the order in which operations are applied becomes important.

We show all invalid interleavings of operations that will violate transactional guarantees (i.e., cause histories to be non-strictly serializable) in Table~\ref{tab:interleavings}.\footnote{We derive these following Schwarz's method~\cite{schwarz} that reasons about invalid histories using \emph{dependencies}. All operations performed by a transaction can be thought of in terms of reads and writes, and these operations create read-write, write-write, etc.\ dependency edges between two transactions. Schwartz asserts that invalid histories must necessarily include cycles in the dependency graph consisting of some number of read-write, write-read, or write-write edges.}
Preventing these interleavings requires adding complexity to each flat combining call (pop and push), as well as adding flat combining calls to check, undo, or install at commit time. We describe several methods to do so, and argue that these methods cannot be integrated with the flat combining algorithm without introducing overhead that reduces its performance to below that of the T-QueueO or T-QueueP. 

\begin{table}
    \centering
    \begin{tabular}{|c|l|}
        \hline
\multicolumn{2}{|c|}{Interleaving}\\
        \hline
1. & 
\begin{lstlisting}
(T1, Q.pop(), true/false)  
(T2, Q.pop(), true/false)       
(T1, Q.pop(), true/false)
\end{lstlisting} 
       \\ 
    \hline
        2. & 
\begin{lstlisting}
// Q.size() == 1  
(T1, Q.pop(), true) // Q empty  
(T2, Q.push(a), ())
(T1, Q.pop(), true)
\end{lstlisting} 
       \\ 
    \hline
        3. & 
\begin{lstlisting}
// Q.size() == 1  
(T1, Q.pop(), true)  // Q empty  
(T2, Q.pop(), false)
(T1, Q.push(a), ())
\end{lstlisting} 
\\
\hline
        4. &
\begin{lstlisting}
(T1, Q.push(a), ()) 
(T2, Q.push(a), ())
(T1, Q.push(a), ())
\end{lstlisting} 
\\
\hline
        5. &
\begin{lstlisting}
// Q.size() == 0 
(T1, Q.push(a), ())       
(T2, Q.pop(), true)  // Q empty
(T1, Q.pop(), false) 
\end{lstlisting} 
\\
    \hline
\end{tabular}
    \caption{Invalid operation interleavings in transactional histories.}
    \label{tab:interleavings}
\end{table}

A standard method for a transactional queue is to delay push operation execution, and to use a pessimistic or optimistic approach when encountering an empty queue.
To prevent interleavings 4 and 5, all pushes are delayed until commit time. These interleavings can occur only if $T1$'s first push is visible to $T2$ prior to $T1$'s commit. If we delay pushes until commit time, $T2$ will not detect the presence of a pushed item in the queue.

Because pop operations immediately return values that depend on the state of the queue (\texttt{false} if the queue is empty or \texttt{true} if the queue is nonempty), interleavings 1, 2, and 3 cannot be prevented by delaying pop operations until commit time. Instead, we can take one of two approaches. Let $T1$ be a transaction that has performed a pop.
\begin{enumerate}
    \item Optimistic: Abort $T1$ at commit time if $T2$ has committed an operation that would cause an invalid interleaving.
    \item Pessimistic: Prevent $T2$ from committing any operation until after $T1$ commits or aborts.
\end{enumerate}

The T-QueueO implements the optimistic method: checks of the tail version and the head version determine at commit time whether the empty status of the queue has been modified by another, already committed transaction. The T-QueueP implements the pessimistic approach, which locks the queue after a pop is performed and only releases the lock if the transaction commits or aborts, therefore preventing any other transaction from committing any operation after the pop.

The flat combining approach can do either approach (1) or (2) to support transactions; however, the flat combining approach cannot do either without introducing overhead that reduces its performance to below that of the T-QueueO or T-QueueP.

If we take approach (1), a pop cannot be performed at execution time because no locks on the queue are acquired at execution time: other transactions are allowed to commit pops, which may pop an invalid head if this transaction aborts. Thus, in order to determine if a pop should return true or return false, a transactional pop flat combining request requires much more complexity than a non-transactional one: the thread must determine how many elements the queue holds, how many elements the current transaction is intending to pop, and if any other thread intends to pop (in which case the transaction aborts). The transactional push flat combining request is also more complex, as it requires installing all the pushes of the transaction. Additional flat combining calls are necessary to allow a thread to perform checks of the queue's empty status (the \texttt{<EMPTY?>} flat combining call) to determine whether the transaction can commit or must abort, and to actually execute the pops at commit time. Thus, approach (1) requires adding both more flat combining calls and more complexity to the existing flat combining calls.

If we take approach (2), the flat combining approach can either perform a pop at execution time or delay the pop until commit time. If the pop is performed at execution time, then the thread must acquire a global lock on the queue after a pop and hold the lock until commit: this prevents another thread from observing an inconsistent state of the queue. If a pop removes the head of the queue prior to commit and the transaction later aborts, the popped element must be re-attached to the head of the queue. Any thread performing a pop must acquire a global lock to ensure that no other thread can commit a transaction that pops off the incorrect head of the queue (given that elements may be reattached to the head if the transaction aborts). Additional flat combining calls are necessary to acquire or release the global lock. 

We can also imagine a mix of approaches (1) and (2). If a transaction $T1$ executes a pop, we can disallow any pops from other transactions (using the equivalent of a global lock) but allow other transactions containing only pushes to commit prior to $T1$ completing. This approach prevents interleavings 1 and 3, but requires performing a check of the queue's empty status, as in approach (1), if a pop saw the queue in an empty state. This is because another transaction may have committed a push between the time of $T1$'s pop and $T1$'s completion. This mixed approach outperforms both approach (2) and approach (1), and is the approach described as the flat combining algorithm in Chapter~\ref{queue}. 

As previously noted, all possible approaches to prevent interleavings 1, 2, and 3 rely on implementing additional flat combining calls and increasing the complexity of previously existing flat combining calls. In addition, acquisition of a global ``lock'' on the queue for approach (2) prevents the combiner thread from applying \emph{all} of the requests it sees; instead, requests will either return ``abort'' to the calling thread or not be applied, leading to additional time spent spinning or repeating requests. Together, these modifications to the flat combining algorithm allow the combiner thread to prevent all invalid transactional history interleavings in Table~\ref{tab:interleavings}.

We see through our experiments that these changes to the flat combining algorithm reduce its performance such that it performs worse than a naive synchronization algorithm; furthermore, we claim that these changes, or changes similar in nature, are necessary in order to provide transactional guarantees. The original flat combining algorithm does not need to synchronize transactions, because the commutativity of singleton transactions is equivalent to the commutativity of single queue operations. Any history of operations in data structures supporting only singleton transactions (i.e., a concurrent, non-transactional data structure) is valid, and the combiner thread is allowed to immediately apply all threads' operation requests in arbitrary order. However, this property that makes flat combining so performant disappears as soon as the algorithm has to deal with transactions that do not commute, and handle invalid, non-serializable histories. In the next section, we demonstrate how changing the queue specification to allow for greater operation commutativity in a transactional setting leads to a version of flat combining that can outperform our T-QueueO and T-QueueP algorithms: this supports our claim that the flat combining algorithm's performance is heavily dependent on the commutativity of the particular queue specification in a transactional setting.
