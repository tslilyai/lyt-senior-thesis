\chapter{Commutativity in Transactional and Weakly Transactional Queue Specifications}
\label{commutativity}

We first describe the commutativity of queue operations in highly-concurrent, non-transactional settings and compare it to the commutativity of queue operations in a transactional setting. We then argue that the flat combining technique, while perhaps near-optimal for a highly-concurrent data structure, is no better for performance than a naive synchronization technique in a transactional data structure. This is because the flat combining algorithm's high performance comes from exploiting the greater operation commutativity present in a non-transactional setting. The flat combining algorithm's optimizations require heavy modifications to support transactions, leading to significant performance loss. 

To demonstrate this point, we propose a transactional specification that allows for greater operation commutativity, with the expectation that the flat combining technique can achieve better performance under this specification. Our experimental results illustrate that the commutativity of operations in this new setting is essential for the flat combining technique to be effective.

\section{Commutativity Terminology and Results}
We introduce some basic terminology (as defined by Schwarz~\cite{schwarz} and Weihl~\cite{weihl}) that will occur in our discussion.

\subsection{Histories}
\begin{defn}
    A \emph{history} is a sequence of \texttt{(transaction, operation, result)} tuples that represent an interleaving of operations of all committed transactions. A history also includes \texttt{(transaction, COMMIT)} tuples that represent the time of the transaction's commit. Knowledge of both the history and initial conditions of a data structure leads to complete knowledge of the (high-level) end state of the structure.

\begin{eg}
    \singlespacing   

    \begin{lstlisting}

    // Q.size() == 0 
    (T2, Q.push(a), ())
    (T1, Q.pop(), true)
    (T2, Q.push(a), ())
    (T1, Q.pop(), true)
    (T1, COMMIT)
    (T2, COMMIT)
    // Final State: Q.size() == 0 
    \end{lstlisting}
    \doublespacing
\end{eg}

\end{defn}

\begin{defn}
    A history $H'$ is \emph{consistent} with $H$ if:
    \begin{enumerate}
        \item $H'$ contains the same tuples as $H$: the same transactions were executed with the same return values for all operations within the transactions.
        \item The order of a single thread's calls in $H'$ remains consistent with the thread's order of calls in $H$.
    \end{enumerate}
\end{defn}

\begin{defn}
    A history $H$ is \emph{serial} if all tuples are grouped by transaction: if $i\le j\le k$ and $H_i$ and $H_k$ are from the same transaction, then $H_j$ is also from that transaction. This means the tuples form a serial transaction order.
\end{defn}
\begin{defn}
    A history $H$ is \emph{serializable} if there exists a serial history $H'$ s.t. $H'$ is consistent with $H$.

\end{defn}

\begin{eg}
    $H$ is a serializable history whose corresponding serial execution is $H'$. $H''$ represents a serial history, but is inconsistent with $H$ because its pop operations return different results.
\begin{figure}[H]
\singlespacing   
   \begin{tabular}{c|c|c}
H & H' & H''\\
\hline
\begin{lstlisting}
// Q.size() == 0 
(T2, Q.push(a), ())
(T1, Q.pop(), true)
(T2, Q.push(a), ())
(T1, Q.pop(), true)
(T1, COMMIT)
(T2, COMMIT)
\end{lstlisting} & 
\begin{lstlisting}
// Q.size() == 0 
(T2, Q.push(a), ())
(T2, Q.push(a), ())
(T1, Q.pop(), true)
(T1, Q.pop(), true)
(T1, COMMIT)
(T2, COMMIT)
\end{lstlisting} &
\begin{lstlisting}
// Q.size() == 0 
(T1, Q.pop(), false)
(T1, Q.pop(), false)
(T2, Q.push(a), ())
(T2, Q.push(a), ()) 
(T1, COMMIT)
(T2, COMMIT)
\end{lstlisting}
\end{tabular}
\end{figure}
\end{eg}

\begin{defn}
A history is \emph{linearizable} if all transactions appears to occur instantaneously between their start time and their commit time: if transaction $T1$ commits before transaction $T2$, then $T1$ must appear before $T2$ in the serial history~\cite{harristm}.
\end{defn}

\begin{defn}
    A history $H$ is \emph{strictly serializable}, or \emph{valid}, if it is both serializable and linearizable. 
\end{defn}

\begin{eg}
$H$ is a serializable, but not linearizable history. This is because $T2$ should have observed the pushes committed by $T1$. We can find a serial ordering of $H$, shown in $H'$, but $H'$ violates the rule that the serial order of transactions corresponds to the real time order of the transactions' commits.
    
\begin{figure}[H]
    \centering
\singlespacing   
    \begin{tabular}{c|c}
H & H'\\
\hline
\begin{lstlisting}
// Q empty                          
(T1, Q.push(a), ())                
(T1, Q.push(a), ())               
(T1, Q.pop(), true)
(T1, COMMIT)
(T2, Q.pop(), false)
(T2, COMMIT)
\end{lstlisting} & 
\begin{lstlisting}
// Q empty
(T2, Q.pop(), false)
(T2, COMMIT)
(T1, Q.push(a), ())                       
(T1, Q.push(a), ())
(T1, Q.pop(), true)
(T1, COMMIT)
\end{lstlisting}
    \end{tabular}
\end{figure}
\end{eg}

\subsection{Dependencies}

Note that all operations can be classified as sets of reads and/or writes (as we do in Table~\ref{table:qrw}). We therefore define dependencies abstractly as reads and writes of particular objects in our definitions.

\begin{defn}
    A \emph{dependency} between transaction $T2$ and transaction $T1$ can be defined as one of the following relations:
    \begin{itemize}
        \item R-R: $T2$ reads an object previously read by $T1$
        \item R-W: $T2$ writes an object previously read by $T1$
        \item W-R: $T2$ reads an object previously written by $T1$
        \item W-W: $T2$ writes an object previously written by $T1$
    \end{itemize}
    Dependencies between two transactions form a dependency graph, where transactions are the vertices and labeled edges indicate different types of dependencies between them.
\end{defn}

\begin{defn}
    Operation $p$ performed by $T1$ \emph{commutes} with operation $q$ of $T2$ when the operations form a \emph{R-R} dependency or no dependency at all~\cite{weihl}.
\end{defn}

\subsection{Commutativity and Serializability Results}

These results are well-known in the literature about transactional data structure scalability; we repeat them here for reference.
A history $H$ is serializable if there are no \emph{cyclic dependencies}---cycles consisting of some number of \emph{R-W}, \emph{W-R}, or \emph{W-W} dependency edges between any two transactions in the dependency graph (see Schwarz's proof~\cite{schwarz}). This implies that when two operations commute, exchanging the order in which they occur in the history does not affect the serializability of the history. From this, it follows that the number of valid histories of a transactional data structure is dependent on the commutativity of its operations.

These results imply that commutativity of a data structure's operations determines the data structure's scalability. In the context of a transactional data structure, a greater number of valid histories leads to lower contention between transactions and greater scalability. 

\section{Commutativity of Concurrent and Transactional Queues}

For generality, we reduce each queue operation to a read or write of a particular semantic object: the \texttt{head}, the \texttt{tail}, or the \texttt{empty?} predicate of the queue. This allows for our reasoning to be applied to push or pop operations that differ from our current specification of pop or push. For example, we can imagine an alternative pop operation that returns \texttt{void} regardless of whether the queue was empty, which would perform no visible reads. We summarize the reads and writes of our pop and push specifications in Table~\ref{table:qrw}.

\begin{table}[t]
\centering
\begin{tabular}{c||c|c}
    Operation & Read & Write\\
    \hline
    pop & \texttt{empty?} & \texttt{head}, \texttt{empty?}\\
    push & & \texttt{tail}, \texttt{empty?}\\
\end{tabular}
    \caption[Objects read or written by queue operations]{Objects read or written by queue operations. A push or pop only performs a write of \texttt{empty?} if it changes the empty status of the queue.}
    \label{table:qrw}
\end{table}

We begin our discussion by examining the dependencies generated by singleton transactions (Table~\ref{tab:queuesimpledeps}) and the cyclic dependencies generated by multi-operation transactions that cause invalid histories (Table~\ref{tab:interleavings}). 
Dependencies generated by singleton transactions indicate which operations need to be synchronized for concurrent correctness: the presence of a non-$R$-$R$ dependency between two operations (i.e., two singleton transactions) indicates that the two operations need to be synchronized for correctness in a concurrent, but non-transactional, setting.
In a transactional setting, we encounter cyclic dependencies that cause invalid histories. To prevent these dependency cycles, a transactional queue algorithm can implement one of several methods, which we describe in detail.
We then demonstrate that, while the flat combining algorithm achieves high performance when correctly synchronizing singleton transactions, integrating the flat combining algorithm with any of the methods to deal with cyclic dependencies greatly reduces its effectiveness.

\subsection{Operation Dependencies of Concurrent, Non-Transactional Queues}

We use the dependencies of singleton transactions to determine when two operations must be synchronized in a concurrent, non-transactional queue. This requires that the concurrent, non-transactional queue satisfies the guarantees of a transactional data structure performing singleton transactions.
The flat combining technique provides serializable, atomic, and linearizable singleton transactions~\cite{flatcombining}. We can therefore use transactional dependencies of singleton transactions to reason about when the concurrent, non-transactional flat combining queue must add mechanisms to ensure correct synchronization.\footnote{We note that, in general, the guarantees of concurrent, non-transactional queues are almost, but not exactly, equivalent to that of singleton transactions. A history of singleton transactions is automatically serializable: the history corresponding to the ordering of operation execution is a serial ordering of transactions. The atomicity of transactions is guaranteed by the correctness properties of the concurrent data structure. However, single operations are not always linearizable: depending on the implementation of the data structure, the effects of a operation $P$ that has ``committed'' (i.e., has returned) may not be visible to an operation that is performed after $P$ returns. An example of this is the Stone concurrent queue~\cite{stone}, in which a slow enqueue may cause a subsequent dequeue to observe an empty queue.}

The dependency relations between singleton transactions are shown in Table~\ref{tab:queuesimpledeps}. 
A concurrent, non-transactional queue cannot have any cyclic dependencies, since such dependencies require that there a transaction can have more than one operation. Therefore, all histories are strictly serializable. Synchronization is only necessary to ensure correctness when two threads attempt to write (\emph{W-W}) the same object, or when one thread writes and another thread simultaneously reads (\emph{W-R} or \emph{R-W}) the same object; this means that synchronizing concurrent access creates the main performance bottleneck. As we have shown, the flat combining approach works particularly well in a highly-concurrent setting to minimize the overhead of this synchronization cost.

\begin{table}[H]
    \singlespace
    \centering
    \begin{tabular}{|l|c|}
        \hline
        \multicolumn{1}{|c|}{Interleaving} & Generated Dependencies\\
        \hline

\begin{lstlisting}
(T1, Q.pop(), true/false) 
(T2, Q.pop(), true/false)
\end{lstlisting} &
W-W, R-W, W-R, R-R
       \\ 
    \hline
\begin{lstlisting}
(T1, Q.push(a), ())
(T2, Q.push(a), ())
\end{lstlisting} &
W-W
\\
\hline
\begin{lstlisting}
(T1, Q.pop(), true/false) 
(T2, Q.push(a), ()) 
\end{lstlisting} &
R-W, W-W
       \\ 
    \hline
\begin{lstlisting}
(T1, Q.push(a), ())
(T2, Q.pop(), true)
\end{lstlisting} &
W-R, W-W
\\
    \hline
\end{tabular}
    \caption[Dependencies generated by singleton transactions that require synchronization.]{Dependencies generated by singleton transactions that require synchronization. A W-W dependency indicates that both transactions wrote the same object; a R-W dependency indicates that T1 read and later T2 wrote the same object; etc.

    A write of \texttt{empty?} occurs only when the operation changes the empty status of the queue; dependencies generated by these writes are included. R-R dependencies are omitted because they do not require synchronization.
}
    \label{tab:queuesimpledeps}
\end{table}

\subsection{Dependency Cycles in Transactional Queues}

\begin{table}
    \centering
    \begin{tabular}{|c|l|c|}
        \hline
\multicolumn{2}{|c|}{Interleaving} & Generated Dependency Cycle\\
        \hline
1. & 
\begin{lstlisting}
(T1, Q.pop(), true/false)  
(T2, Q.pop(), true/false)       
(T1, Q.pop(), true/false)
\end{lstlisting} &
W-W-W, R-W-W, W-R-W, R-W-R, W-W-R 
       \\ 
    \hline
        2. & 
\begin{lstlisting}
// Q.size() == 1  
(T1, Q.pop(), true) // Q empty  
(T2, Q.push(a), ())
(T1, Q.pop(), true)
\end{lstlisting} &
R-W-R
       \\ 
    \hline
        3. & 
\begin{lstlisting}
// Q.size() == 1  
(T1, Q.pop(), true)  // Q empty  
(T2, Q.pop(), false)
(T1, Q.push(a), ())
\end{lstlisting} &
W-W-W, W-R-W
\\
\hline
        4. &
\begin{lstlisting}
(T1, Q.push(a), ()) 
(T2, Q.push(a), ())
(T1, Q.push(a), ())
\end{lstlisting} &
W-W-W
\\
\hline
        5. &
\begin{lstlisting}
// Q.size() == 0 
(T1, Q.push(a), ())       
(T2, Q.pop(), true)  // Q empty
(T1, Q.pop(), false) 
\end{lstlisting} &
W-R-W, W-W-W
\\
    \hline
\end{tabular}
    \caption[Dependency cycles generated by multi-operation transactions that cause invalid histories.]{Dependency cycles generated by multi-operation transactions that cause invalid histories. 
    A W-W-W cycle indicates that T1 first wrote to an object $o$, then T2 wrote to $o$, and then T1 again wrote $o$; A R-W-W cycle indicates that T1 first read an object $o$, then T2 wrote to $o$, and then T1 wrote to $o$; etc.
    
    A write of \texttt{empty?} occurs only when the operation changes the empty status of the queue. We show only those specific scenarios that encounter empty queues and therefore generate writes, creating problematic dependencies cycles. Cycles that contain R-R dependencies are omitted.}
    \label{tab:interleavings}
\end{table}

Cyclic dependencies can occur in a transactional queue supporting multi-operation transactions, thus reducing the number of valid histories. The possible interleavings that generate cyclic dependencies that cause invalid histories are shown in Table~\ref{tab:interleavings}. Nearly all of the interleavings that result in cyclic dependencies, and therefore invalid histories, occur when the queue becomes empty. 
We only see cyclic dependencies that cause invalid histories in a nonempty queue if both transactions push or both transactions pop (a \emph{W-W} dependency).

\lyt{XXX CHECK}
Cyclic dependencies add an additional level of dependencies that the algorithm must handle. An algorithm must still correctly handle dependencies between operations, but now must also handle the extra cyclic dependencies that form between transactions. This requires, as indicated by the transactional commit protocol, that a transactional algorithm provide methods to perform the operation both at execution and at install time, to later check that it did not observe something that another transaction did that would make its operation results invalid, and to undo any eagerly-made updates if it must abort. 

For flat combining, this means that its fundamental principle---allowing any thread to act as if it is any other thread---cannot be upheld. Tracking dependencies between transactions requires the combiner thread to have knowledge (in some abstract sense) of which thread and which transaction each request on the publication list belongs to. It cannot treat every push or pop request in the publication list as equal to every other: to do so means the combiner thread will blindly apply operations in interleavings that create the invalid histories shown in Table~\ref{tab:interleavings}.
Introducing this knowledge while retaining the publication list-combiner thread technique requires teaching the combiner thread about transactions. Concretely, this means introducing additional complexity to each flat combining call (pop or push); introducing new flat combining calls to check, undo, or install at commit time; or a combination of both. 
We describe several methods to do so below, and argue that these methods cannot be integrated with the flat combining algorithm without introducing overhead that reduces its performance to below that of the T-QueueO or T-QueueP.

There are several methods for preventing cyclic dependencies from forming (i.e., preventing invalid operation interleavings from occurring in the history). For example, a standard method for a transactional queue is to delay push operation execution, and to use a pessimistic or optimistic approach when encountering an empty queue. This method can prevent all invalid interleavings shown in Table~\ref{tab:interleavings}.

To prevent interleavings 4 and 5, all pushes are delayed until commit time. These interleavings can occur only if $T1$'s first push is visible to $T2$ prior to $T1$'s commit. If we delay pushes until commit time, $T2$ will not detect the presence of a pushed item in the queue.

Because pop operations immediately return values that depend on the state of the queue (\texttt{false} if the queue is empty or \texttt{true} if the queue is nonempty), interleavings 1, 2, and 3 cannot be prevented by delaying pop operations until commit time. Instead, we can take one of two approaches. Let $T1$ be a transaction that has performed a pop.
\begin{enumerate}
    \item Optimistic: Abort $T1$ at commit time if $T2$ has committed an operation that would cause an invalid interleaving.
    \item Pessimistic: Prevent $T2$ from committing any operation until after $T1$ commits or aborts.
\end{enumerate}

The T-QueueO implements the optimistic method: checks of the tail version and the head version determine at commit time whether the empty status of the queue has been modified by another, already committed transaction. The T-QueueP implements the pessimistic approach, which locks the queue after a pop is performed and only releases the lock if the transaction commits or aborts, therefore preventing any other transaction from committing any operation after the pop.

The flat combining approach can do either approach (1) or (2) to support transactions; however, the flat combining approach cannot do either without introducing overhead that reduces its performance to below that of the T-QueueO or T-QueueP.

If we take approach (1), a pop cannot be performed at execution time because no locks on the queue are acquired at execution time: other transactions are allowed to commit pops, which may pop an invalid head if this transaction aborts. Thus, in order to determine if a pop should return true or return false, a transactional pop flat combining request requires much more complexity than a non-transactional one: the thread must determine how many elements the queue holds, how many elements the current transaction is intending to pop, and if any other thread intends to pop (in which case the transaction aborts). The transactional push flat combining request is also more complex, as it requires installing all the pushes of the transaction. Additional flat combining calls are necessary to allow a thread to perform checks of the queue's empty status (the \texttt{<EMPTY?>} flat combining call) to determine whether the transaction can commit or must abort, and to actually execute the pops at commit time. Thus, approach (1) requires adding both more flat combining calls and more complexity to the existing flat combining calls.

If we take approach (2), the flat combining approach can either perform a pop at execution time or delay the pop until commit time. If the pop is performed at execution time, then the thread must acquire a global lock on the queue after a pop and hold the lock until commit: this prevents another thread from observing an inconsistent state of the queue. If a pop removes the head of the queue prior to commit and the transaction later aborts, the popped element must be re-attached to the head of the queue. Any thread performing a pop must acquire a global lock to ensure that no other thread can commit a transaction that pops off the incorrect head of the queue (given that elements may be reattached to the head if the transaction aborts). Additional flat combining calls are necessary to acquire or release the global lock. 

We can also imagine a mix of approaches (1) and (2). If a transaction $T1$ executes a pop, we can disallow any pops from other transactions (using the equivalent of a global lock) but allow other transactions containing only pushes to commit prior to $T1$ completing. This approach prevents interleavings 1 and 3, but requires performing a check of the queue's empty status, as in approach (1), if a pop saw the queue in an empty state. This is because another transaction may have committed a push between the time of $T1$'s pop and $T1$'s completion. This mixed approach outperforms both approach (2) and approach (1), and is the approach described as the flat combining algorithm in Chapter~\ref{queue}. 

As previously noted, all possible approaches to prevent interleavings 1, 2, and 3 rely on implementing additional flat combining calls and increasing the complexity of previously existing flat combining calls. In addition, acquisition of a global ``lock'' on the queue for approach (2) prevents the combiner thread from applying \emph{all} of the requests it sees; instead, requests will either return ``abort'' to the calling thread or not be applied, leading to additional time spent spinning or repeating requests. These modifications to the flat combining algorithm are, in essence, giving the combiner thread the necessary knowledge about transactions to prevent invalid interleavings.

We see through our experiments that these changes to the flat combining algorithm reduce its performance such that it performs worse than a naive synchronization algorithm; furthermore, we claim that these changes, or changes similar in nature, are necessary in order to provide transactional guarantees. The original flat combining algorithm exploits the property that any correct history of operations in data structures supporting only singleton transactions (i.e., a concurrent, non-transactional data structure) is valid. The combiner thread is allowed to immediately apply all threads' operation requests in arbitrary order. However, this property that makes flat combining so performant disappears as soon as the algorithm has to deal with invalid histories. In the next section, we demonstrate how ignoring specific invalid histories leads to a version of flat combining that can outperform our T-QueueO and T-QueueP algorithms: this supports our claim that the flat combining algorithm's performance is heavily dependent on what types of transactional guarantees it must provide.
