\section{The Weakly Transactional Queue (QueueWT)}

The Weakly Transactional Queue (QueueWT) demonstrates how the flat combining technique's performance is dependent upon the number of invalid histories. This queue implements a weaker transactional specification, which provides all invariants of a concurrent queue, but provides the following guarantees instead of the transactional ones listed earlier (Chapter~\ref{Queue}).
\begin{itemize}
    \item Within a transaction, any two \texttt{pops} do not need to pop consecutive values off the queue.
    \item A \texttt{pop} cannot remove an uninstalled value (i.e., a value \texttt{push}ed earlier in the same transaction).
\end{itemize}

Given this specification, interleavings 1, 2, and 3 in Table~\ref{tab:interleavings} are allowable because two \texttt{pop}s in a transaction do not need to pop consecutive values off the queue. Interleaving 4 is prevented by installing all \texttt{push}es in the same transaction together at commit time, and interleaving 5 is allowed because $T1$'s \texttt{pop} cannot see its earlier \texttt{push}ed value.

\subsection{Algorithm}

\subsection{Evaluation}
%\input{figures/wtxnal_queues.fig}

We evaluate the weakly transactional flat-combining queue on the same benchmarks described in Section \ref{microbenchmarks} to compare against the fully transactional flat-combining queue and our other STO queues.

\subsubsection{2-Thread Push-Pop Test}

\subsubsection{Multi-Thread Random Singleton Transactions Test}

\subsubsection{Multi-Thread Random Multi-Operation Transactions Test}

\subsection{Discussion}
